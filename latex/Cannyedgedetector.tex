\documentclass[12pt,a4paper,twoside,openright,italian]{book}
\usepackage{babel}
\usepackage[ansinew]{inputenc} 
\usepackage[T1]{fontenc}
\usepackage[safe]{textcomp}
\usepackage{calc}
\usepackage{setspace}
\onehalfspacing
\usepackage[paper=a4paper,twoside=true]{geometry}
\geometry{lmargin=4cm,rmargin=2.5cm,tmargin=3cm,bmargin=2cm}
\geometry{headheight=15pt}%se no fancyhdr rompe le scatole
\geometry{dvips}

%HEADERS
\usepackage{fancyhdr}%da mettere sempre dopo cambi a textwidth
\fancypagestyle{corpo}{\fancyhf{}%
\pagestyle{fancy}%
\fancyhead{}%
\fancyhead[RO]{\rightmark}%
\fancyhead[LE]{\leftmark}%
\fancyfoot{}%
\fancyfoot[CO,CE]{\thepage}%
\renewcommand\headrulewidth{0.4pt}}
\fancypagestyle{indice}{\fancyhf{}%
\pagestyle{fancy}%
\fancyhead{}%
\fancyhead[LO,RE]{\rightmark}%
\fancyhead[RO,LE]{\thepage}
\fancyfoot{}%
%\fancyfoot[CO,CE]{\thepage}%
\renewcommand\headrulewidth{0.4pt}}
\fancypagestyle{conclusione}{\fancyhf{}%
\pagestyle{fancy}%
\fancyhead{}%
%\fancyhead[LO,RE]{\rightmark}%
%\fancyhead[RO,LE]{\thepage}
\fancyfoot{}%
\fancyfoot[CO,CE]{\thepage}%
\renewcommand\headrulewidth{0.4pt}}

%IMPORT FIGURE, usare jpg con pdf e eps con dvi!
\usepackage[dvips]{graphicx}

%MATEMATICA
\usepackage{amsfonts}	%font matematici
\usepackage{amsmath}	%miglioramenti visuali per i font
\usepackage{amssymb}	%altri simboli matematici non standard

%RIFERIMENTI A FIGURE E SEZIONI
\usepackage[italian]{varioref}
\labelformat{figure}{figura~#1}

%RIFERIMENTI BIBLIOGRAFICI
\usepackage{url}
\usepackage{cite}%forse non necessario

%CAPTION DEI FLOAT
\usepackage[font={it},labelfont={bf,up}]{caption}

%ASPETTO DEI LISTATI DI CODICE
\usepackage{color}
\usepackage{listings}
\lstloadlanguages{C++, java}
\lstset{language={Java}}
%per avere gli accenti
\lstset{extendedchars=true}
%\lstset{keywordstyle=\color{blue},basicstyle=\ttfamily\small}
\lstset{keywordstyle=\color{red}\bfseries,basicstyle=\ttfamily\small}
%\lstset{ndkeywordstyle=\color{red}\bfseries,basicstyle=\ttfamily\small}
\lstset{showstringspaces=false}
\definecolor{grey}{rgb}{.8,.8,.8}
\definecolor{darkgrey}{rgb}{.9,.9,.9}
\lstset{commentstyle=\color{grey}\itshape}
%\lstset{stringstyle=\color{darkgrey}
\lstset{columns=flexible,tabsize=2}
\lstset{aboveskip=0pt,belowskip=0pt}
%eventualmente mettere caption e label per ogni listato
 
\begin{document}

\frontmatter

%titolo
\begin{titlepage}
\begin{center}
\large{\textsc{Universit� di Padova -- Facolt� di ingegneria}}
\\
\LARGE{\textbf{Corso di laurea in Ingegneria Informatica}}
\\
\vspace{2cm}
\Huge{\textbf{CANNY EDGE DETECTOR}}
\end{center}
\begin{flushleft}

\vspace{5cm}
\Large{Professori: Pagello Enrico, Menegatti Emanuele}
\\
\vspace{1cm}
\Large{Studente: Maglie Andrea, matr. 456188}
\\
\vspace{1.5cm}
\vspace{2cm}
%\Large{Padova, data}
\end{flushleft}
\vspace{3cm}
\begin{center}
\Large{Anno Accademico 2005/2006 }
\end{center}
\end{titlepage}

%%SISTEMARE HEADINGS SBALLATI
\pagestyle{indice}
\tableofcontents
%\listoffigures

\mainmatter
\pagestyle{corpo}
%bisogna eliminare i fancyheaders


\chapter[Canny Edge Detector]{Canny Edge Detector}

Nel 1986 John Canny individu� tre importanti caratteristice che un edge detector deve avere:
\begin{enumerate}
 \item \textbf{Tasso di errore} - l'edge detector deve restituire solo i lati, e dovrebbe trovarli tutti.
 \item \textbf{Localizzazione} - la distanza tra i pixel del lato trovato e del lato reale deve essere la minima possibile.
 \item \textbf{Risposta} - l'edge detector non deve identificare pi� lati quando ne esiste solo uno.
\end{enumerate}
L'edge detector � un filtro di convoluzione che smussa il rumore e localizza i lati; l'obiettivo � identificare il filtro che ottimizza i tre criteri sopra riportati. Ci� pu� essere raggiunto cercando il filtro che massimizza il prodotto \textbf{SNR*localizzazione}, dove SNR � il rapporto segnale/rumore e la localizzazione � un valore che rappresenta il reciproco della distanza del lato localizzato dal lato reale. Una buona approssimazione � data dalla derivata prima della funzione gaussiana:

\begin{equation}
G(x)=e^{-\frac{x^2}{2{\sigma}^2}}
\end{equation}

\begin{equation}
G'(x)=\left(-\frac{x}{{\sigma}^2}\right)e^{\left(-\frac{x^2}{2\sigma ^2}\right)}
\end{equation}
In due dimensioni la gaussiana � data da:
\begin{equation}
G(x,y)={\sigma}^2e^{\left(-\frac{x^2+y^2}{2{\sigma}^2}\right)}
\end{equation}
la quale ha derivate sia lungo \(x\) che lungo \(y\). Facendo la convoluzione tra un'immagine e \(G'\) otteniamo un'immagine con i lati evidenziati, anche in presenza di rumore.
La convoluzione con una gaussiana a due dimensioni pu� essere scomposta in due convoluzioni con una gaussiana unidimensionale. 
In conclusione, l'algoritmo di Canny consiste nei seguenti passi:
\begin{enumerate}
	\item Leggere l'immagine \(I\) che dovr� essere processata.
	\item Creare una maschera gaussiana unidimensionale \(G\) da convolvere con \(I\). La deviazione standard � un parametro dell'algoritmo.
	\item Creare una maschera unidimensionale per la derivata prima della gaussiana nelle direzioni \(x\) e \(y\), ottenendo \(G_x\) e \(G_y\).
	\item Calcolare la convoluzione dell'immagine \(I\) con \(G\) lungo le righe per ottenere la componente \(I_x\) dell'immagine lungo \(x\), e lungo le colonne per ottenere la componente \(I_y\) lungo \(y\).
	\item Calcolare la convoluzione di \(I_x\) con \(G_x\) per ottenere \(I_x'\), cio� la componente lungo \(x\) di \(I\) convoluta con la derivata della gaussiana, e la convoluzione di \(I_y\) con \(G_y\) per ottenere \(I_y'\).
	\item Il modulo di ogni pixel dell'immagine risultante � dato da:
		\begin{equation}
		M(x,y)=\sqrt{I_x'(x,y)^2+I_y'(x,y)^2}
		\end{equation}
	\item Applicare la ``non-maxima suppression'', cio� la rimozione dei pixel che non costituiscono massimi locali. Infatti il modulo del gradiente del pixel di un lato � maggiore del modulo del gradiente dei pixel adiacenti e non facenti parti del lato.
	\item Applicare la sogliatura con isteresi. Vengono usate una soglia alta \(T_h\) e una bassa \(T_l\): ogni pixel avente un valore maggiore di \(T_h\) � assunto come facente parte di un lato; successivamente, ogni pixel collegato al precedente avente un valore pi� grande di \(T_l\) viene assunto come facente parte del lato.
\end{enumerate}

\chapter[Codice sorgente]{Codice sorgente}

\section[Canny.java]{Canny.java}
\lstinputlisting{Canny.java}

\section[CannyShow.java]{CannyShow.java}
\lstinputlisting{CannyShow.java}

\chapter{Esempi}

L'applicazione viene avviata tramite il comando \texttt{java CannyShow}; l'interfaccia grafica � riportata in \ref{fig:esempio1}. Come si pu� vedere, i parametri impostabili sono: dimensione del kernel, soglia superiore, soglia inferiore e deviazione standard. Una volta caricato il file immagine e impostati i parametri desiderati, � sufficente cliccare sul pulsante \emph{Avvia processo} per iniziare la computazione dell'algoritmo di Canny. Non appena la computazione � completa, l'immagine risultante verr� visualizzata nella parte destra (indicata con \emph{Immagine in output}).
In \ref{fig:esempio2} c'� un esempio di utilizzo dell'applicazione con i seguenti parametri:
\begin{itemize}
	\item Kernel: 3x3
	\item Soglia Superiore: 50
	\item Soglia Inferiore: 20
	\item Deviazione Standard: 1.0
\end{itemize}
Nell'esempio di \ref{fig:esempio3} sono stati utilizzati invece i seguenti valori per i parametri:
\begin{itemize}
	\item Kernel: 10x10
	\item Soglia Superiore: 50
	\item Soglia Inferiore: 20
	\item Deviazione Standard: 1.5
\end{itemize}
Ulteriori esempi sono riportati in \ref{fig:esempio4}, \ref{fig:esempio5} e \ref{fig:esempio6}.


\begin{figure}
	\centering
		\includegraphics[width=1.0\textwidth]{esempio1.jpg}
		\caption{}
	\label{fig:esempio1}
\end{figure}
\begin{figure}
	\centering
		\includegraphics[width=1.0\textwidth]{esempio2.jpg}
		\caption{}
	\label{fig:esempio2}
\end{figure}
\begin{figure}
	\centering
		\includegraphics[width=1.0\textwidth]{esempio3.jpg}
		\caption{}
	\label{fig:esempio3}
\end{figure}
\begin{figure}
	\centering
		\includegraphics[width=1.0\textwidth]{esempio4.jpg}
		\caption{Parametri utilizzati: kernel 3x3, soglia superiore 100, soglia inferiore 50, deviazione standard 1.0. }
	\label{fig:esempio4}
\end{figure}
\begin{figure}
	\centering
		\includegraphics[width=1.0\textwidth]{esempio5.jpg}
		\caption{Parametri utilizzati: kernel 10x10, soglia superiore 200, soglia inferiore 100, deviazione standard 0.45. }
	\label{fig:esempio5}
\end{figure}
\begin{figure}
	\centering
		\includegraphics[width=1.0\textwidth]{esempio6.jpg}
		\caption{Parametri utilizzati: kernel 7x7, soglia superiore 50, soglia inferiore 10, deviazione standard 1.6. }
	\label{fig:esempio6}
\end{figure}



\backmatter
%\include{conclusione}
\pagestyle{corpo}
%carica e configura la bibliografia
%\bibliographystyle{plain_ita}
%\addtocontents{toc}{\protect\vspace{2ex}}   %%These two lines are needed in
%\addcontentsline{toc}{chapter}{Bibliografia}%%order to have a the bibliography
%\bibliography{bibliografia}
\nocite{*}
\end{document}